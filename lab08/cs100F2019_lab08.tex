\documentclass[11pt]{article}

% NOTE: The "Edit" sections are changed for each assignment

% Edit these commands for each assignment

\newcommand{\assignmentduedate}{November 20}
\newcommand{\assignmentassignedate}{November 13}
\newcommand{\assignmentnumber}{Eight}

\newcommand{\labyear}{2018}
\newcommand{\labday}{Tuesday}
\newcommand{\labtime}{2:30 pm}

\newcommand{\assigneddate}{Assigned: \labday, \assignmentassignedate, \labyear{} at \labtime{}}
\newcommand{\duedate}{Due: \labday, \assignmentduedate, \labyear{} at \labtime{}}

% Edit these commands to give the name to the main program

\newcommand{\mainprogram}{\lstinline{CommandLineCalculator}}
\newcommand{\mainprogramsource}{\lstinline{src/main/java/labeight/CommandLineCalculator.java}}

\newcommand{\secondprogram}{\lstinline{NumericalCalculator}}
\newcommand{\secondprogramsource}{\lstinline{src/main/java/labeight/NumericalCalculator.java}}

% Edit this commands to describe key deliverables

\newcommand{\reflection}{\lstinline{writing/reflection.md}}
\newcommand{\readme}{\lstinline{README.md}}

% Commands to describe key development tasks

% --> Running gatorgrader.sh
\newcommand{\gatorgraderstart}{\command{gradle grade}}
\newcommand{\gatorgradercheck}{\command{gradle grade}}

% --> Compiling and running program with gradle
\newcommand{\gradlebuild}{\command{gradle build}}
\newcommand{\gradlerun}{\command{gradle run}}

% Commands to describe key git tasks

\newcommand{\gitcommitfile}[1]{\command{git commit #1}}
\newcommand{\gitaddfile}[1]{\command{git add #1}}

\newcommand{\gitadd}{\command{git add}}
\newcommand{\gitcommit}{\command{git commit}}
\newcommand{\gitpush}{\command{git push}}
\newcommand{\gitpull}{\command{git pull}}

\newcommand{\gitaddmainprogram}{\command{git add src/main/java/labeight/CommandLineCalculator.java}}
\newcommand{\gitcommitmainprogram}{\command{git commit src/main/java/labeight/CommandLineCalculator.java -m "Your
descriptive commit message"}}

% Use this when displaying a new command

\newcommand{\command}[1]{``\lstinline{#1}''}
\newcommand{\program}[1]{\lstinline{#1}}
\newcommand{\url}[1]{\lstinline{#1}}
\newcommand{\channel}[1]{\lstinline{#1}}
\newcommand{\option}[1]{``{#1}''}
\newcommand{\step}[1]{``{#1}''}

\usepackage{pifont}
\newcommand{\checkmark}{\ding{51}}
\newcommand{\naughtmark}{\ding{55}}

\usepackage{listings}
\lstset{
  basicstyle=\small\ttfamily,
  columns=flexible,
  breaklines=true
}

\usepackage{fancyhdr}
\usepackage{fancyvrb}

\usepackage[margin=1in]{geometry}
\usepackage{fancyhdr}

\pagestyle{fancy}

\fancyhf{}
\rhead{Computer Science 111}
\lhead{Laboratory Assignment \assignmentnumber{}}
\rfoot{Page \thepage}
\lfoot{\duedate}

\usepackage{titlesec}
\titlespacing\section{0pt}{6pt plus 4pt minus 2pt}{4pt plus 2pt minus 2pt}

\newcommand{\labtitle}[1]
{
  \begin{center}
    \begin{center}
      \bf
      CMPSC 111\\Introduction to Computer Science I\\
      Fall 2017\\
      \medskip
    \end{center}
    \bf
    #1
  \end{center}
}

\begin{document}

\thispagestyle{empty}

\labtitle{Laboratory \assignmentnumber{} \\ \assigneddate{} \\ \duedate{}}

\section*{Objectives}

In this laboratory assignment, you will learn more about using iteration
constructs to perform numerical calculations, further explore the creation of
formatted output, practice calling methods in another Java class, practice using
conditional logic, and learn how to use {\tt while} loops, {\tt ArrayLists}, and
{\tt Iterator}s. You will create a program, comprised of two Java classes, that
performs calculations for the arithmetic mean and the minimum and maximum
values. You will also learn how to modify the inputs so that you can extensively
test your program and establish a confidence in its correctness.

\section*{Suggestions for Success}

\begin{itemize}
  \setlength{\itemsep}{0pt}

\item {\bf Use the laboratory computers}. The computers in this laboratory feature specialized software for completing
  this course's laboratory and practical assignments. If it is necessary for you to work on a different machine, be sure
  to regularly transfer your work to a laboratory machine so that you can check its correctness. If you cannot use a
  laboratory computer and you need help with the configuration of your own laptop, then please carefully explain its
  setup to a teaching assistant or the course instructor when you are asking questions.

\item {\bf Follow each step carefully}. Slowly read each sentence in the assignment sheet, making sure that you
  precisely follow each instruction. Take notes about each step that you attempt, recording your questions and ideas and
  the challenges that you faced. If you are stuck, then please tell a teaching assistant or instructor what assignment
  step you recently completed.

\item {\bf Regularly ask and answer questions}. Please log into Slack at the start of a laboratory or practical session
  and then join the appropriate channel. If you have a question about one of the steps in an assignment, then you can
  post it to the designated channel. Or, you can ask a student sitting next to you or talk with a teaching assistant or
  the course instructor.

\item {\bf Store your files in GitHub}. As in the past laboratory assignments, you will be responsible for storing all
  of your files (e.g., Java source code and Markdown-based writing) in a Git repository using GitHub Classroom. Please
  verify that you have saved your source code in your Git repository by using \command{git status} to ensure that
  everything is updated. You can see if your assignment submission meets the established correctness requirements by
  using the provided checking tools on your local computer and by checking the commits in GitHub.

\item {\bf Keep all of your files}. Don't delete your programs, output files, and written reports after you submit them
  through GitHub; you will need them again when you study for the quizzes and examinations and work on the other
  laboratory, practical, and final project assignments.

\item {\bf Back up your files regularly}. All of your files are regularly backed-up to the servers in the Department of
  Computer Science and, if you commit your files regularly, stored on GitHub. However, you may want to use a flash
  drive, Google Drive, or your favorite backup method to keep an extra copy of your files on reserve. In the event of
  any type of system failure, you are responsible for ensuring that you have access to a recent backup copy of all your
  files.

\item {\bf Explore teamwork and technologies}. While certain aspects of the laboratory assignments will be challenging
  for you, each part is designed to give you the opportunity to learn both fundamental concepts in the field of computer
  science and explore advanced technologies that are commonly employed at a wide variety of companies. To explore and
  develop new ideas, you should regularly communicate with your team and/or the teaching assistants and tutors.

\item {\bf Hone your technical writing skills}. Computer science assignments require to you write technical
  documentation and descriptions of your experiences when completing each task. Take extra care to ensure that your
  writing is interesting and both grammatically and technically correct, remembering that computer scientists must
  effectively communicate and collaborate with their team members and the tutors, teaching assistants, and course
  instructor.

\item {\bf Review the Honor Code on the syllabus}. While you may discuss your assignments with others, copying source
  code or writing is a violation of Allegheny College's Honor Code.

\end{itemize}

\section*{Reading Assignment}

After reviewing the GitHub materials, all of the assignment sheets for the past
laboratory and practical sessions, and the course slides and notes, you should
study Sections 5.1 through 5.8 of your textbook. As in a past assignment, you
should also review the textbook's diagrams that are in Figures 4.7 and 4.8. See
the course instructor if you have any questions about these readings.

\section*{Accessing the Laboratory Assignment on GitHub}

To access the laboratory assignment, you should go into the
\channel{\#announcements} channel in our Slack team and find the announcement
that provides a link for it. Copy this link and paste it into your web browser.
Now, you should accept the laboratory assignment and see that GitHub Classroom
created a new GitHub repository for you to access the assignment's starting
materials and to store the completed version of your assignment. Specifically,
to access your new GitHub repository for this assignment, please click the green
``Accept'' button and then click the link that is prefaced with the label ``Your
assignment has been created here''. If you accepted the assignment and correctly
followed these steps, you should have created a GitHub repository with a name
like
``Allegheny-Computer-Science-100-Fall-2018/computer-science-100-fall-2018-lab-8-gkapfham''.
Unless you provide the instructor with documentation of the extenuating
circumstances that you are facing, not accepting the assignment means that you
automatically receive a failing grade for it.

Before you move to the next step of this assignment, please make sure that you
read all of the content on the web site for your new GitHub repository, paying
close attention to the technical details about the commands that you will type
and the output that your program must produce. Now you are ready to download the
starting materials to your laboratory computer. Click the ``Clone or download''
button and, after ensuring that you have selected ``Clone with SSH'', please
copy this command to your clipboard. To enter into the right directory you
should now type \command{cd cs100F2018}. Next, you can type the command
\command{ls} and see that there are some files or directories inside of this
directory. By typing \command{git clone} in your terminal and then pasting in
the string that you copied from the GitHub site you will download all of the
code for this assignment. For instance, if the course instructor ran the
\command{git clone} command in the terminal, it would look like:

\begin{lstlisting}
  git clone git@github.com:Allegheny-Computer-Science-100-F2018/computer-science-100-fall-2018-lab-7-gkapfham.git
\end{lstlisting}

After this command finishes, you can use \command{cd} to change into the new
directory. If you want to \step{go back} one directory from your current
location, then you can type the command \command{cd ..}. Please continue to use
the \command{cd} and \command{ls} commands to explore the files that you
automatically downloaded from GitHub. What files and directories do you see?
What do you think is their purpose? Spend some time exploring, sharing your
discoveries with a neighbor and a \mbox{teaching assistant}.

\begin{figure}[tb]
\begin{Verbatim}[commandchars=\\\{\}]
  Gregory M. Kapfhammer Fri Oct 26 09:49:50 EDT 2018
  Welcome to the Command Line Calculator!

  Okay, I read in 5 values. Here they are:

  25
  125
  19
  -10
  154

  Nice, I found the minimum value: -10
  Great, I found the maximum value: 154
  Okay, I calculated the arithmetic mean as: 62.6

  Thanks for using the Command Line Calculator.
\end{Verbatim}
\vspace*{-.1in}
\caption{Sample ``{\tt CommandLineCalculator}'' output featuring numerical values read from a file.}
\label{fig:output}
\end{figure}

\section*{Extending and Implementing a Numerical Computation Engine}

You should explore your repository by using a text editor to study the source
code of the {\tt CommandLineCalculator.java} and {\tt NumericalCalculator.java}
files. What methods do these classes provide? How do they work? Can you draw a
picture that depicts the relationship between these two Java classes? Do you see
how one class will call the methods of the other? Before you start the next step
of this assignment, make sure that you understand what parts of the system are
already implemented and what you have to extend and add in order to complete the
assignment.

You will notice that this assignment organizes the methods into two separate
classes, as you have seen in past assignments and in-class exercises. In
particular, the {\tt CommandLineCalculator} provides the user interface for our
program and the {\tt NumericalCalculator} furnishes the methods that perform the
required computations. If you want to make changes to the way in which the
program accepts input or produces output, then you will need to modify the {\tt
CommandLineCalculator}. Otherwise, if you want to modify the way in which the
program performs a computation, or add a new computation, then you must make
changes to the {\tt NumericalCalculator}. Overall, these two Java classes
complete their work by following a pattern similar to that which is outlined in
Figures 4.7 and 4.8 of your textbook. Please see the instructor if you have
questions about this approach.

Additionally, it is important to note that this assignment asks you to finish
segments in both the {\tt CommandLineCalculator.java} and {\tt
NumericalCalculator.java} files. For instance, step three and steps five through
nine are not finished in the \mainprogram{}. This means that you will have to
refer to previous assignments to learn more about how to read input from a file.
Then, to perform computations you will invoke all of the methods in the
\secondprogram{}. Yet, the \secondprogram{} also contains several methods in
need of a finished implementation. Please refer to the \readme{} file in your
repository for all of the requirements associated with these files. Finally, you
should review the partial implementation of the \program{public static double
calculateArithmeticMean(ArrayList<Integer> numberList)} method as it provides an
illustrative example of how to iterate through the \program{numberList} in order
to perform a numerical computation.

\section*{Checking the Correctness of Your Program and Writing}

As verified by Checkstyle, the code for the \mainprogramsource{} file must
adhere to all of the requirements in the Google Java Style Guide available at
\url{https://google.github.io/styleguide/javaguide.html}. The Markdown file that
contains your reflection must adhere to the standards described in the Markdown
Syntax Guide \url{https://guides.github.com/features/mastering-markdown/}.
Finally, your \reflection{} file should adhere to the Markdown standards
established by the \step{Markdown linting} tool available at
\url{https://github.com/markdownlint/markdownlint/} and the writing standards
set by the \step{prose linting} tool from \url{http://proselint.com/}. Instead
of requiring you to manually check that your deliverables adhere to these
industry-accepted standards, the GatorGrader tool that you will use in this
laboratory assignment makes it easy for you to automatically check if your
submission meets these well-established standards for correctness. Please see
the instructor if you have questions about GatorGrader.

Since this is not your first laboratory assignment, you will notice that the
provided source code does not contain all of the required comments at the top of
the Java source code file. This means that you will have to inspect the source
code from previous laboratory and practical assignments to review how to create
the comments in the \mainprogramsource{} file. Moreover, the provided source
code is missing many of the lines that are needed to pass the GatorGrader
checks. Review the requirements for these Java code files, as outlined in the
previous section. You can study the source code of that file to learn more about
what you need to add to it. Don't forget to look in your GitHub repository to
learn about GatorGrader's many checks!

To get started with the use of GatorGrader, type the command \gatorgraderstart{}
in your terminal. If your laboratory work does not meet all of the assignment's
requirements, then you will see a summary of the failing checks along with a
statement giving the percentage of checks that are currently passing. If you do
have mistakes in your assignment, then you will need to review GatorGrader's
output, find the mistake, and try to fix it. Once your program is building
correctly, fulfilling at least some of the assignment's requirements, you should
transfer your files to GitHub using the \gitcommit{} and \gitpush{} commands.
For example, if you want to signal that the \mainprogramsource{} file has been
changed and is ready for transfer to GitHub you would first type
\gitcommitmainprogram{} in your terminal, followed by typing \gitpush{} and then
checking to see that the transfer to GitHub was successful. Make sure that you
transfer your source code and technical writing to GitHub, as the instructor
cannot access your deliverables unless you run the \gitpush{} command. Please
see the instructor if you cannot upload your deliverables.

After the course instructor enables \step{continuous integration} with a system
called Travis CI, when you use the \gitpush{} command to transfer your source
code to your GitHub repository, Travis CI will initialize a \step{build} of your
assignment, checking to see if it meets all of the requirements. If both your
source code and writing meet all of the established requirements, then you will
see a green \checkmark{} in the listing of commits in GitHub after awhile. If
your submission does not meet the requirements, a red \naughtmark{} will appear
instead. The instructor will reduce a student's grade for this assignment if the
red \naughtmark{} appears on the last commit in GitHub immediately before the
assignment's due date. Yet, if the green \checkmark{} appears on the last commit
in your GitHub repository, then you satisfied all of the main checks, thereby
allowing the course instructor to evaluate other aspects of your source code and
writing, as further described in the \step{Evaluation} section of this
assignment sheet. Unless you provide the instructor with documentation of the
extenuating circumstances that you are facing, no late work will be considered
towards your grade for this laboratory assignment. In conclusion, here are some
points to remember for creating programs that performs computations:

\begin{enumerate}
  \setlength{\itemsep}{0pt}

\item The provide source code is incomplete---make sure that you add all of the needed features.

\item You should draw a technical diagram to show the relationships between the Java classes.

\item Make sure that you review the textbook's diagrams that explain parameter
  passing in Java.

\item Make sure to carefully practice implementing iteration constructs and
  conditional logic.

\item As in past assignments, your program only needs to have one {\tt main} method in one file.

\item Your program will alternate between creating and displaying textual output---this is okay!

\item Your program should support reading in a variable number of data values
  from the provided text file. To ensure its correctness, make sure that you
  test the program with different inputs! For instance, you should try a
  different number of inputs and different numerical values.

\item Don't forget to review the assignment sheets from the previous laboratory
  and practical assignments as they contain insights that will support your
  completion of this assignment.

\end{enumerate}

\section*{Summary of the Required Deliverables}

\noindent Students do not need to submit printed source code or technical writing for any assignment in this course.
Instead, this assignment invites you to submit, using GitHub, the following deliverables.

\begin{enumerate}

  \setlength{\itemsep}{0in}

\item Stored in \reflection{}, a one-paragraph reflection on the commands that
  you typed in a text editor and the terminal window. This Markdown-based
  document should explain the input, output, and behavior of each command and
  the challenges that you confronted when using it. For every challenge that you
  encountered, please explain your solution for it. This document should also
  explain how you fixed each of the defects in the provided code.

\item A complete and correct version of \mainprogramsource{} that meets all of
  the set requirements and produces the desired textual output in the terminal.

\item A complete and correct version of \secondprogramsource{} that meets all of
  the set requirements and supports the desired textual output in the terminal.

\end{enumerate}

\section*{Evaluation of Your Laboratory Assignment}

Using a report that the instructor shares with you through the commit log in
GitHub, you will privately received a grade on this assignment and feedback on
your submitted deliverables. Your grade for the assignment will be a function of
the whether or not it was submitted in a timely fashion and if your program
received a green \checkmark{} indicating that it met all of the requirements.
Other factors will also influence your final grade on the assignment. In
addition to studying the efficiency and effectiveness of your Java source code,
the instructor will also evaluate the accuracy of both your technical writing
and the comments in your source code. If your submission receives a red
\naughtmark{}, the instructor will reduce your grade for the assignment while
still considering the regularity with which you committed to your GitHub
repository and the overall quality of your partially completed work. Please see
the instructor if you have questions about the evaluation of this laboratory
assignment.

It is understood that an important part of the learning process in any course,
and particularly one in computer science, derives from thoughtful discussions
with teachers and fellow students. Such dialogue is encouraged. However, it is
necessary to distinguish carefully between the student who discusses the
principles underlying a problem with others and the student who produces
assignments that are identical to, or merely variations on, someone else's work.
While it is acceptable for students in this class to discuss their programs,
data sets, and reports with their classmates, deliverables that are nearly
identical to the work of others will be taken as evidence of violating the
\mbox{Honor Code}. You must uphold the Honor Code while completing this
assignment. Students who do not understand how to adhere to the Honor Code
should talk with the course instructor during office hours.

\end{document}
