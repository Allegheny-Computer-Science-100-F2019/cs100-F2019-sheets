\documentclass[11pt]{article}

% NOTE: The "Edit" sections are changed for each assignment

% Edit these commands for each assignment

\newcommand{\assignmentduedate}{November 18}
\newcommand{\assignmentassignedate}{November 15}
\newcommand{\assignmentnumber}{Eight}

\newcommand{\labyear}{2019}
\newcommand{\labdueday}{Monday}
\newcommand{\labassignday}{Friday}
\newcommand{\labtime}{1:30 pm}

\newcommand{\assigneddate}{Assigned: \labassignday, \assignmentassignedate, \labyear{} at \labtime{}}
\newcommand{\duedate}{Due: \labdueday, \assignmentduedate, \labyear{} at \labtime{}}

% Edit these commands to give the name to the main program

\newcommand{\mainprogram}{\lstinline{PasswordCheckerMain}}
\newcommand{\mainprogramsource}{\lstinline{src/main/java/practicaleight/PasswordCheckerMain.java}}

\newcommand{\secondprogram}{\lstinline{PasswordChecker}}
\newcommand{\secondprogramsource}{\lstinline{src/main/java/practicaleight/PasswordChecker.java}}

% Edit this commands to describe key deliverables

\newcommand{\reflection}{\lstinline{writing/reflection.md}}

% Commands to describe key development tasks

% --> Running gatorgrader.sh
\newcommand{\gatorgraderstart}{\command{gradle grade}}
\newcommand{\gatorgradercheck}{\command{gradle grade}}

% --> Compiling and running program with gradle
\newcommand{\gradlebuild}{\command{gradle build}}
\newcommand{\gradlerun}{\command{gradle run}}

% Commands to describe key git tasks

% NOTE: Could be improved, problems due to nesting

\newcommand{\gitcommitfile}[1]{\command{git commit #1}}
\newcommand{\gitaddfile}[1]{\command{git add #1}}

\newcommand{\gitadd}{\command{git add}}
\newcommand{\gitcommit}{\command{git commit}}
\newcommand{\gitpush}{\command{git push}}
\newcommand{\gitpull}{\command{git pull}}

\newcommand{\gitcommitmainprogram}{\command{git commit src/main/java/practicaleight/PasswordCheckerMain.java -m "Your
descriptive commit message"}}

% Use this when displaying a new command

\newcommand{\command}[1]{``\lstinline{#1}''}
\newcommand{\program}[1]{\lstinline{#1}}
\newcommand{\url}[1]{\lstinline{#1}}
\newcommand{\channel}[1]{\lstinline{#1}}
\newcommand{\option}[1]{``{#1}''}
\newcommand{\step}[1]{``{#1}''}

\usepackage{pifont}
\newcommand{\checkmark}{\ding{51}}
\newcommand{\naughtmark}{\ding{55}}

\usepackage{listings}
\lstset{
  basicstyle=\small\ttfamily,
  columns=flexible,
  breaklines=true
}

\usepackage{fancyvrb}
\usepackage{color}

\usepackage{fancyhdr}

\usepackage[margin=1in]{geometry}
\usepackage{fancyhdr}

\pagestyle{fancy}

\fancyhf{}
\rhead{Computer Science 100}
\lhead{Practical Assignment \assignmentnumber{}}
\rfoot{Page \thepage}
\lfoot{\duedate}

\usepackage{titlesec}
\titlespacing\section{0pt}{6pt plus 4pt minus 2pt}{4pt plus 2pt minus 2pt}

\newcommand{\labtitle}[1]
{
  \begin{center}
    \begin{center}
      \bf
      CMPSC 100\\Computational Expression\\
      Fall 2019\\
      \medskip
    \end{center}
    \bf
    #1
  \end{center}
}

\begin{document}

\thispagestyle{empty}

\labtitle{Practical \assignmentnumber{} \\ \assigneddate{} \\ \duedate{}}

\section*{Objectives}

To continue practicing the use of GitHub to access the files for a practical
assignment. Additionally, to practice using the Ubuntu operating system and
software development programs such as a ``Docker container'' and the ``Atom text
editor''. You will continue to practice using Slack to support communication
with the teaching assistants and the course instructor. Next, you will create
and then call methods that determine if certain events occurred during the year
input from the file. For this task you will use {\tt if/else} statements and
boolean logic operators such as ``{\tt \&\&}'' or ``{\tt ||}''. Finally, you
will continue to learn more about creating and using Java classes and objects as
you create a password checker that a systems administrator could use to ensure
computer security.

\section*{Suggestions for Success}

\begin{itemize}
  \setlength{\itemsep}{0pt}

\item {\bf Follow each step carefully}. Slowly read each sentence in the
  assignment sheet, making sure that you precisely follow each instruction.
  Take notes about each step that you attempt, recording your questions and
  ideas and the challenges that you faced. If you are stuck, then please tell a
  technical leader or instructor what assignment step you recently completed.

\item {\bf Regularly ask and answer questions}. Please log into Slack at the
  start of a laboratory or practical session and then join the appropriate
  channel. If you have a question about one of the steps in an assignment, then
  you can post it to the designated channel. Or, you can ask a student sitting
  next to you or talk with a teaching assistant or the course instructor.

\item {\bf Store your files in GitHub}. As in all of your past assignments, you
  will be responsible for storing all of your files (e.g., Java source code and
  Markdown-based writing) in a Git repository using GitHub Classroom. Please
  verify that you have saved your source code in your Git repository by using
  \command{git status} to ensure that everything is updated. You can see if
  your assignment submission meets the established correctness requirements by
  using the provided checking tools on your local computer and in checking the
  commits in GitHub.

\item {\bf Keep all of your files}. Don't delete your programs, output files,
  and written reports after you submit them through GitHub; you will need them
  again when you study for the quizzes and examinations and work on the other
  laboratory, practical, and final project assignments.

\item {\bf Explore teamwork and technologies}. While certain aspects of these
  assignments will be challenging for you, each part is designed to give you the
  opportunity to learn both fundamental concepts in the field of computer
  science and explore advanced technologies that are commonly employed at a wide
  variety of companies. To explore and develop new ideas, you should regularly
  communicate with your team and/or the student technical leaders.

\item {\bf Hone your technical writing skills}. Computer science assignments
  require to you write technical documentation and descriptions of your
  experiences when completing each task. Take extra care to ensure that your
  writing is interesting and both grammatically and technically correct,
  remembering that computer scientists must effectively communicate and
  collaborate with their team members and the tutors, teaching assistants, and
  course instructor.

\item {\bf Review the Honor Code on the syllabus}. While you may discuss your
  assignments with others, copying source code or writing is a violation of
  Allegheny College's Honor Code.

\end{itemize}

\section*{Reading Assignment}

If you have not done so already, please read all of the relevant ``GitHub
Guides'', available at \url{https://guides.github.com/}, that explain how to use
many of the GitHub's features. In particular, please make sure that you have
read guides such as ``Mastering Markdown'' and ``Documenting Your Projects on
GitHub''; each of them will help you to understand how to use both GitHub and
GitHub Classroom. Focusing on the content about creating and using Java objects
and writing conditional logic, you should review Chapters 1 through 4 and
Sections 5.1 and 5.8 in the textbook.

\section*{Implementing and Evaluating a Password Checker}

To access the practical assignment, you should go into the
\channel{\#announcements} channel in our Slack team and find the announcement
that provides a link for it. Copy this link and paste it into your web browser.
Now, you should accept the practical assignment and see that GitHub Classroom
created a new GitHub repository for you to access the assignment's starting
materials and to store the completed version of your assignment. Specifically,
to access your new GitHub repository for this assignment, please click the green
``Accept'' button and then click the link that is prefaced with the label ``Your
assignment has been created here''. If you accepted the assignment and correctly
followed these steps, you should have created a GitHub repository with a name
like
``Allegheny-Computer-Science-100-Fall-2019/computer-science-100-fall-2019-practical-8-gkapfham''.
Unless you provide the instructor with documentation of the extenuating
circumstances that you are facing, not accepting the assignment means that you
automatically receive a failing grade for it. Please follow the steps from the
previous laboratory assignments for finding your ``home base'' for this
practical assignment; see the instructor if you are stuck on getting started.

This practical assignment invites you to create a program that will check an
input password. In one of the classes for this assignment, {\tt
PasswordChecker.java}, you will write methods that, for the file's input from
{\tt PasswordCheckerMain.java}, determine if the chosen password:

\begin{itemize}

\item Contains ten or more characters

\item Contains at least one upper-case and lower-case letter

\item Contains at least one number

\end{itemize}

Figure~\ref{mad} contains the output from running a program like the one you
must implement. You should study the comments in the \mainprogramsource{} to see
each step that you have to implement. You should also look at the
\secondprogramsource{} to see the methods that will ultimately contain the
conditional logic. After finishing the both of these files, you should
repeatedly test you program to make sure that it is creating the correct textual
output. This will involve you editing the input file and then building and
running the program and checking the output to ensure that it produces different
values and that the checks are correct. Don't forget that this assignment
requires you to understand and edit two different files called
\mainprogramsource{} and \secondprogramsource{}. This means that you must have
correct formatting and documentation in both of these files; check the {\tt
README.md} file for a statement of other checks. You should also be able to draw
a diagram explaining the relationship between the two files.

\begin{figure}[tb]
\begin{Verbatim}[commandchars=\\\{\}]
Gregory M. Kapfhammer Fri Nov 15 12:41:12 EST 2019

I will read in a password from a file.
Okay, I read in the password "computerscience2019".

Is the password "computerscience2019" a valid password? No

Thank you for using the PasswordChecker.
\end{Verbatim}
\vspace*{-.1in}
\caption{Sample ``{\tt PasswordCheckerMain}'' output featuring output from conditional logic checks.}
\label{mad}
\end{figure}

When you are testing your \mainprogramsource{}, please make sure that you try
both valid and invalid passwords. For instance, the output in Figure~\ref{mad}
shows how the program should work when you input an invalid password. What does
the program's output look like when you input a valid password? Finally, you
should notice that there are several ways in which you could enhance this
password checker. For example, while the program currently accepts a single
password, it would be more useful if it could read and check a list of
passwords. Although it is not a requirement of this assignment, you should think
about how to add this feature. Moreover, the output in Figure~\ref{mad} shows
that the program currently does not explain why the password is invalid. Again,
even though it is not a requirement for this assignment, you should think about
how to enhance the {\tt PasswordCheckerMain} so that it can report precisely why
the password is not valid. Please see the instructor with questions about how to
add these extra features.
%
Finally, as you are working on the \program{isValidPassword()} method please
consider this code as a starting point.

\begin{verbatim}
    boolean isValidLength = false;
    boolean isValidCapitalized = false;
    boolean isValidLowerCase = false;
    boolean isValidNumber = false;
    int passwordIndex = 0;

    // check the length
    if (password.length() > PASSWORD_MINIMUM_LENGTH) {
      isValidLength = true;
    }
\end{verbatim}

\section*{Checking the Correctness of Your Program and Writing}

As in the past assignments, you are provided with an automated tool for checking
the quality of your source code. To check your Java source code you can started
with the use of GatorGrader, type the command \gatorgraderstart{} in your Docker
container. If your work does not meet all of the assignment's requirements, then
you will see output that highlights the mistakes that you made. If you do have
mistakes in your assignment, then you will need to review GatorGrader's output,
find the mistake, and try to fix it. Specifically, don't forget to add in the
required comments and commit to your repository a sufficient number of times!
This assignment also requires that you resolve and remove all of the
\command{TODO} markers in both of the provided Java source code files. This will
often involve you taking source code from the text book and adding it to the
right location in the provided files. Finally, don't forget to ensure that you
still transfer all of your source code to your GitHub repository so that it is
available for assessment. Please see the course instructor if you have questions
about any of these important steps for mastering the technical concepts.

Once your program is building correctly, fulfilling at least some of the
assignment's requirements, you should transfer your files to GitHub using the
\gitcommit{} and \gitpush{} commands. For example, if you want to signal that
the \mainprogramsource{} file has been changed and is ready for transfer to
GitHub you would first type \gitcommitmainprogram{} in your terminal, followed
by typing \gitpush{} and checking to see that the transfer to GitHub is
successful. If you notice that transferring your code to GitHub did not work
correctly, then please try to determine why, asking a teaching assistant or the
course instructor for help, if necessary.

After the course instructor enables \step{continuous integration} with a system
called Travis CI, when you use the \gitpush{} command to transfer your source
code to your GitHub repository, Travis CI will initialize a \step{build} of your
assignment, checking to see if it meets all of the requirements. If both your
source code and writing meet all of the established requirements, then you will
see a green \checkmark{} in the listing of commits in GitHub after awhile. If
your submission does not meet the requirements, a red \naughtmark{} will appear
instead. The instructor will reduce a student's grade for this assignment if the
red \naughtmark{} appears on the last commit in GitHub immediately before the
assignment's due date. Yet, if the green \checkmark{} appears on the last commit
in your GitHub repository, then you satisfied all of the main checks. Unless you
provide the course instructor with documentation of the extenuating
circumstances that you are facing, no late work will be considered towards your
completion grade for this practical assignment. You should aim to finish
practical assignments on the day that they are assigned; please see the
instructor if you do not understand this policy.

Since this is another challenging practical assignment and you are still
learning how to use the Java classes and objects, don't become frustrated if you
make a mistake. Instead, use your mistakes as an opportunity for learning both
about the necessary technology and the background and expertise of the other
students in the class, the teaching assistants, and the course instructor.

\noindent Students do not need to submit printed source code or technical
writing for any assignment in this course. Instead, this assignment invites you
to submit, using GitHub, the following deliverables. Because this is a practical
assignment, you are not required to complete any technical writing.

\begin{enumerate}

\setlength{\itemsep}{0in}

\item A correct version of \mainprogramsource{} that meets all of the
  established source code requirements and produces the desired text-based
  output.

\item A correct version of \secondprogramsource{} that meets all of the
  established source code requirements and produces the desired text-based
  output.

\end{enumerate}

\section*{Evaluation of Your Practical Assignment}

Practical assignments are graded on a completion --- or ``checkmark'' --- basis.
If your GitHub repository has a \checkmark{} for the last commit before the
deadline then you will receive the highest possible grade for the assignment.
However, you will fail the assignment if you do not complete it correctly, as
evidenced by a red \naughtmark{} in your commit listing, by the set deadline for
finishing the project.

\section*{Adhering to the Honor Code}

In adherence to the Honor Code, students should complete this practical
assignment on an individual basis. While it is appropriate for students in this
class to have high-level conversations about the assignment, it is necessary to
distinguish carefully between the student who discusses the principles
underlying a problem with others and the student who produces assignments that
are identical to, or merely variations on, someone else's work. Deliverables
(e.g., the Java source code) that are nearly identical to the work of others
will be taken as evidence of violating the \mbox{Honor Code}. Please see the
course instructor if you have questions about this policy.

\end{document}
