\documentclass[11pt]{article}

% NOTE: The "Edit" sections are changed for each assignment

% Edit these commands for each assignment

\newcommand{\assignmentduedate}{November 12}
\newcommand{\assignmentassignedate}{November 5}
\newcommand{\assignmentnumber}{Seven}

\newcommand{\labyear}{2019}
\newcommand{\labday}{Tuesday}
\newcommand{\labtime}{2:30 pm}

\newcommand{\assigneddate}{Assigned: \labday, \assignmentassignedate, \labyear{} at \labtime{}}
\newcommand{\duedate}{Due: \labday, \assignmentduedate, \labyear{} at \labtime{}}

% Edit these commands to give the name to the main program

\newcommand{\mainprogram}{\lstinline{CommandLineGeometer}}
\newcommand{\mainprogramsource}{\lstinline{src/main/java/labseven/CommandLineGeometer.java}}

\newcommand{\secondprogram}{\lstinline{GeometricCalculator}}
\newcommand{\secondprogramsource}{\lstinline{src/main/java/labseven/GeometricCalculator.java}}

% Edit this commands to describe key deliverables

\newcommand{\reflection}{\lstinline{writing/reflection.md}}

% Commands to describe key development tasks

% --> Running gatorgrader.sh
\newcommand{\gatorgraderstart}{\command{gradle grade}}
\newcommand{\gatorgradercheck}{\command{gradle grade}}

% --> Compiling and running program with gradle
\newcommand{\gradlebuild}{\command{gradle build}}
\newcommand{\gradlerun}{\command{gradle run}}

% Commands to describe key git tasks

\newcommand{\gitcommitfile}[1]{\command{git commit #1}}
\newcommand{\gitaddfile}[1]{\command{git add #1}}

\newcommand{\gitadd}{\command{git add}}
\newcommand{\gitcommit}{\command{git commit}}
\newcommand{\gitpush}{\command{git push}}
\newcommand{\gitpull}{\command{git pull}}

\newcommand{\gitaddmainprogram}{\command{git add src/main/java/labseven/CommandLineGeometer.java}}
\newcommand{\gitcommitmainprogram}{\command{git commit src/main/java/labseven/CommandLineGeometer.java -m "Your
descriptive commit message"}}

% Use this when displaying a new command

\newcommand{\command}[1]{``\lstinline{#1}''}
\newcommand{\program}[1]{\lstinline{#1}}
\newcommand{\url}[1]{\lstinline{#1}}
\newcommand{\channel}[1]{\lstinline{#1}}
\newcommand{\option}[1]{``{#1}''}
\newcommand{\step}[1]{``{#1}''}

\usepackage{pifont}
\newcommand{\checkmark}{\ding{51}}
\newcommand{\naughtmark}{\ding{55}}

\usepackage{listings}
\lstset{
  basicstyle=\small\ttfamily,
  columns=flexible,
  breaklines=true
}

\usepackage{fancyhdr}

\usepackage[margin=1in]{geometry}
\usepackage{fancyhdr}

\pagestyle{fancy}

\fancyhf{}
\rhead{Computer Science 100}
\lhead{Laboratory Assignment \assignmentnumber{}}
\rfoot{Page \thepage}
\lfoot{\duedate}

\usepackage{titlesec}
\titlespacing\section{0pt}{6pt plus 4pt minus 2pt}{4pt plus 2pt minus 2pt}

\newcommand{\labtitle}[1]
{
  \begin{center}
    \begin{center}
      \bf
      CMPSC 100\\Computational Expression\\
      Fall 2019\\
      \medskip
    \end{center}
    \bf
    #1
  \end{center}
}

\begin{document}

\thispagestyle{empty}

\labtitle{Laboratory \assignmentnumber{} \\ \assigneddate{} \\ \duedate{}}

\section*{Objectives}

In this laboratory assignment, you will learn more about using the {\tt
java.lang.Math} class to perform numerical calculations, further explore the
creation of formatted output, learn how to use enumerated types, and practice
calling methods in another Java class. Additionally, since real-world software
developers often have to debug source code created by other developers and add
features to existing code, you will participate in a ``bug hunt'' and add new
source code to an existing system. Ultimately, you will create a working
program, comprised of two Java classes, that performs numerical calculations
through a command-line interface run in the terminal window.

\section*{Suggestions for Success}

\begin{itemize}
  \setlength{\itemsep}{0pt}

\item {\bf Follow each step carefully}. Slowly read each sentence in the
  assignment sheet, making sure that you precisely follow each instruction. Take
  notes about each step that you attempt, recording your questions and ideas and
  the challenges that you faced. If you are stuck, then please tell a technical
  leader or instructor what assignment step you recently completed.

\item {\bf Regularly ask and answer questions}. Please log into Slack at the
  start of a laboratory or practical session and then join the appropriate
  channel. If you have a question about one of the steps in an assignment, then
  you can post it to the designated channel. Or, you can ask a student sitting
  next to you or talk with a technical leader or the course instructor.

\item {\bf Store your files in GitHub}. Starting with this laboratory
  assignment, you will be responsible for storing all of your files (e.g., Java
  source code and Markdown-based writing) in a Git repository using GitHub
  Classroom. Please verify that you have saved your source code in your Git
  repository by using \command{git status} to ensure that everything is
  updated. You can see if your assignment submission meets the established
  correctness requirements by using the provided checking tools on your local
  computer and by checking the commits in GitHub.

\item {\bf Keep all of your files}. Don't delete your programs, output files,
  and written reports after you submit them through GitHub; you will need them
  again when you study for the quizzes and examinations and work on the other
  laboratory, practical, and final project assignments.

\item {\bf Explore teamwork and technologies}. While certain aspects of the
  laboratory assignments will be challenging for you, each part is designed to
  give you the opportunity to learn both fundamental concepts in the field of
  computer science and explore advanced technologies that are commonly employed
  at a wide variety of companies. To explore and develop new ideas, you should
  regularly communicate with your team and/or the student technical leaders.

\item {\bf Hone your technical writing skills}. Computer science assignments
  require to you write technical documentation and descriptions of your
  experiences when completing each task. Take extra care to ensure that your
  writing is interesting and both grammatically and technically correct,
  remembering that computer scientists must effectively communicate and
  collaborate with their team members and the student technical leaders and
  course instructor.

\item {\bf Review the Honor Code on the syllabus}. While you may discuss your
  assignments with others, copying source code or writing is a violation of
  Allegheny College's Honor Code.

\end{itemize}

\section*{Reading Assignment}

After reviewing the GitHub materials, all of the assignment sheets for the past
laboratory and practical sessions, and the course slides and notes, you should
review Sections 3.5 through 3.8 of your textbook. To enhance your understanding
of some points in this lab you should additionally examine Figures 4.7 and 4.8.
Specifically, please make sure that you understand how the Java programming
language supports method invocations and performs parameter passing. Please see
the course instructor if you have any questions about this reading assignment.

\section*{Accessing the Laboratory Assignment on GitHub}

To access the laboratory assignment, you should go into the
\channel{\#announcements} channel in our Slack team and find the announcement
that provides a link for it. Copy this link and paste it into your web browser.
Now, you should accept the laboratory assignment and see that GitHub Classroom
created a new GitHub repository for you to access the assignment's starting
materials and to store the completed version of your assignment. Specifically,
to access your new GitHub repository for this assignment, please click the green
``Accept'' button and then click the link that is prefaced with the label ``Your
assignment has been created here''. If you accepted the assignment and correctly
followed these steps, you should have created a GitHub repository with a name
like
``Allegheny-Computer-Science-100-Fall-2019/computer-science-100-fall-2019-lab-7-gkapfham''.
Unless you provide the instructor with documentation of the extenuating
circumstances that you are facing, not accepting the assignment means that you
automatically receive a failing grade for it.

Before you move to the next step of this assignment, please make sure that you
read all of the content on the web site for your new GitHub repository, paying
close attention to the technical details about the commands that you will type
and the output that your program must produce. Now you are ready to download the
starting materials to your laboratory computer. Click the ``Clone or download''
button and, after ensuring that you have selected ``Clone with SSH'', please
copy this command to your clipboard. To enter into the right directory you
should now type \command{cd cs100F2019}. Next, you can type the command
\command{ls} and see that there are some files or directories inside of this
directory. By typing \command{git clone} in your terminal and then pasting in
the string that you copied from the GitHub site you will download all of the
code for this assignment. For instance, if the course instructor ran the
\command{git clone} command in the terminal, it would look like:

\begin{lstlisting}
  git clone git@github.com:Allegheny-Computer-Science-100-F2019/computer-science-100-fall-2019-lab-7-gkapfham.git
\end{lstlisting}

After this command finishes, you can use \command{cd} to change into the new
directory. If you want to \step{go back} one directory from your current
location, then you can type the command \command{cd ..}. Please continue to use
the \command{cd} and \command{ls} commands to explore the files that you
automatically downloaded from GitHub. What files and directories do you see?
What do you think is their purpose? Spend some time exploring, sharing your
discoveries with a neighbor and a \mbox{technical leader}. You should study the
code with an eye towards determining what parts of it are likely to contain
defects!

\section*{Participate in a Realistic ``Bug Hunt''}

You should explore your repository by using a text editor to study the source
code of the {\tt CommandLineGeometer.java} and {\tt GeometricCalculator.java}
files. What methods do these classes provide? How do they work? Does any of this
code look incorrect? Why? After carefully reviewing the source code and PP 3.6,
3.7, and 3.9 on page 145 of your textbook and then compiling and running the
{\tt CommandLineGeometer} class, you should notice that there are several
defects in this program. As such, you will need to take part in a ``bug hunt''
to find and fix all of the problems. First, you should find the method
responsible for calculating the volume of a sphere as given in the following
code segment. Using the equation in PP 3.6 as a reference point, what is the
defect in this method? Please find and fix this defect before moving on to the
next steps in the assignment!

\vspace*{-.05in}
\begin{verbatim}
  public static double calculateSphereVolume(double radius) {
    double volume;
    volume = (3 / 4) * (Math.PI) * radius * radius * radius;
    return volume;
  }
\end{verbatim}
\vspace*{-.05in}

The {\tt GeometricCalculator} also provides a method to calculate a triangle's
area. Once again, there is a mistake in this method. Can you find and fix it?
How did you know that this was the bug? If you investigate the source code of
the method for calculating the volume of a cylinder, you will notice that there
is another defect lurking in the source code. Wait! If you carefully study the
way in which the {\tt CommandLineGeometer} calls the method provided by the {\tt
GeometricCalculator} and then displays the resulting output, you will realize
that there is another bug in this program. Make sure you have found all of the
problems before continuing with this laboratory assignment. Don't forget that
you must document how you fixed the bug with comments in the code and written
content in your reflection document. Please take the necessary time to develop a
thorough understanding of how a Java program's source code might lead to
incorrect output in the terminal.

% You and your group member(s) should complete this assignment by using GitHub and
% proceeding incrementally. That is, one member of the team should be in charge of
% typing when you find and fix the first defect. Then, that person should commit
% and push all of the repaired Java files and the other team member(s) should run
% \command{git pull}. Now, a new team member should perform the typing when
% finding and fixing the next defect. Importantly, one of the goals for this
% assignment is for all of the team members to gain experience with using GitHub
% to collaborate in a team. Please see the course instructor if you do not
% understand how to use GitHub to complete this task.

\section*{Extending the Geometry Calculator}

\begin{sloppypar}
  After reviewing the aforementioned programming projects on
  page 145 of your textbook, you will also notice that the {\tt
  GeometricCalculator.java} does not contain methods for calculating the surface
  area of a sphere or a cylinder. While avoiding all of the mistake types that
  you corrected in the previous phase of this assignment, please add in new
  methods to perform these calculations. In addition, you will need to add
  appropriate input and output statements and method calls to the {\tt
  CommandLineGeometer} to ensure that the entire program works correctly. For
  instance, you will need to implement a new method called {\tt
  calculateSphereSurfaceArea} to the {\tt GeometricCalculator.java} file and
  then add input and output code to the {\tt CommandLineGeometer.java} file.
  Whenever possible, try to follow the correct pattern established in the given
  source code. Please see the course instructor if you have questions!
\end{sloppypar}

As you continue to critically review the source code of the {\tt
CommandLineGeometer}, you will notice that it does not always consistently
produce output for the user. For example, even though it displays the
user-input radius before calling {\tt calculateSphereVolume}, it does not
appropriately display the sides of the triangle for the {\tt
computeTriangleArea} method---can you please add in this feature? Moreover,
none of the output of the {\tt double} variables in the {\tt
CommandLineGeometer} is formatted in a consistent fashion. To solve this
problem, you should use one of the techniques described in Section 3.6 of your
textbook to format the output of all decimals to contain four decimal points.
For instance, you may consider creating an instance of the {\tt DecimalFormat}
class.

Please notice that most of the provided code is not commented. As part of this
assignment, you should add detailed comments to the code, following the JavaDoc
standard as appropriate. Please see the instructor if you are not sure how or
where to add comments to your program. Again, you should also include comments
that explain each of the defects and how you found and fixed them.

\section*{Exploring the Features of the Java Programming Language}

The {\tt CommandLineGeometer} program uses an enumerated type, as described in
Section 3.7, to store specific values in a variable. Intuitively, an enumerated
type allows a variable to take on one of a pre-specified set of values or
levels. In this case, the {\tt GeometricShape} enumerated type can take on three
possible values. What are they? Why is it useful to use this type of variable?

You will notice that this laboratory assignment organizes the methods into two
separate classes, as you have seen in past assignments and in-class exercises.
In particular, the {\tt CommandLineGeometer} provides the user interface for our
program and the {\tt GeometricCalculator} furnishes the methods that perform the
required computations.  If you want to make changes to the way in which the
program accepts input or produces output, then you will need to modify the {\tt
CommandLineGeometer}. Otherwise, if you want to modify the way in which the
program performs a computation, or add a new computation, then you must make
changes to the {\tt GeometricCalculator}. Overall, these two Java classes
complete their work by following a pattern similar to that which is outlined in
Figures 4.7 and 4.8 of your textbook. Please see the instructor if you have
questions about this approach.

Additionally, it is important to note that this assignment asks you to add new
methods to the {\tt GeometricCalculator.java} file. To complete this task, you
should directly copy the pattern that you see in the provided methods, only
making changes to implement the new functionality. Also, these methods accept
parameters, of type {\tt double}, that are passed from the {\tt main} method in
the {\tt CommandLineGeometer} to one of the ``calculate'' methods in the {\tt
GeometricCalculator}. You should also notice that all of the methods return a
{\tt double} variable to the method that calls it. Intuitively, the parameters
are the ``input'' to a method and the return values are the ``output'' of the
method. When you create the required new methods, you should follow the pattern
of the previously implemented method, ensuring that you have the same types of
input and output.

As you are completing this assignment, make sure that you can answer the
following questions:

\begin{enumerate}

  \item How does the Java programming language support parameter passing
    between methods?

  \item What are formal parameters? What are actual parameters? Why do these
    parameters \mbox{exist}?

  \item What are the benefits and drawbacks associated creating separate
    Java methods?

  \item What is an enumerated type and how does this Java program use one of
    these entities?

  \item Why are there no output statements inside of the \secondprogram?

  \item Why are all of the output statements inside of the \mainprogram?

  \item Can you describe the type, details, and purpose of the \program{Math.PI}
    variable?

\end{enumerate}

\section*{Checking the Correctness of Your Program and Writing}

As verified by Checkstyle, the code for the \mainprogramsource{} file must
adhere to all of the requirements in the Google Java Style Guide available at
\url{https://google.github.io/styleguide/javaguide.html}. The Markdown file that
contains your reflection must adhere to the standards described in the Markdown
Syntax Guide \url{https://guides.github.com/features/mastering-markdown/}.
Finally, your \reflection{} file should adhere to the Markdown standards
established by the \step{Markdown linting} tool available at
\url{https://github.com/markdownlint/markdownlint/} and the writing standards
set by the \step{prose linting} tool from \url{http://proselint.com/}. Instead
of requiring you to manually check that your deliverables adhere to these
industry-accepted standards, the GatorGrader tool that you will use in this
laboratory assignment makes it easy for you to automatically check if your
submission meets these well-established standards for correctness. Please see
the instructor if you have questions about GatorGrader.

Since this is not your first laboratory assignment, you will notice that the
provided source code does not contain all of the required comments at the top of
the Java source code file. This means that you will have to inspect the source
code from previous laboratory and practical assignments to review how to create
the comments in the \mainprogramsource{} file. Moreover, the provided source
code is missing many of the lines that are needed to pass the GatorGrader
checks. Review the requirements for these Java code files, as outlined in the
previous section. You can study the source code of this file to learn more about
what you need to add to it. Don't forget to look in your GitHub repository to
learn about GatorGrader's checks!

To get started with the use of GatorGrader, type the command \gatorgraderstart{}
in your terminal. If your laboratory work does not meet all of the assignment's
requirements, then you will see a summary of the failing checks along with a
statement giving the percentage of checks that are currently passing. If you do
have mistakes in your assignment, then you will need to review GatorGrader's
output, find the mistake, and try to fix it. Once your program is building
correctly, fulfilling at least some of the assignment's requirements, you should
transfer your files to GitHub using the \gitcommit{} and \gitpush{} commands.
For example, if you want to signal that the \mainprogramsource{} file has been
changed and is ready for transfer to GitHub you would first type
\gitcommitmainprogram{} in your terminal, followed by typing \gitpush{} and then
checking to see that the transfer to GitHub was successful. Make sure that you
transfer your source code and technical writing to GitHub, as the instructor
cannot access your deliverables unless you run the \gitpush{} command. Please
see the instructor if you cannot upload your deliverables.

After the course instructor enables \step{continuous integration} with a system
called Travis CI, when you use the \gitpush{} command to transfer your source
code to your GitHub repository, Travis CI will initialize a \step{build} of your
assignment, checking to see if it meets all of the requirements. If both your
source code and writing meet all of the established requirements, then you will
see a green \checkmark{} in the listing of commits in GitHub after awhile. If
your submission does not meet the requirements, a red \naughtmark{} will appear
instead. The instructor will reduce a student's grade for this assignment if the
red \naughtmark{} appears on the last commit in GitHub immediately before the
assignment's due date. Yet, if the green \checkmark{} appears on the last commit
in your GitHub repository, then you satisfied all of the main checks, thereby
allowing the course instructor to evaluate other aspects of your source code and
writing, as further described in the \step{Evaluation} section of this
assignment sheet. Unless you provide the instructor with documentation of the
extenuating circumstances that you are facing, no late work will be considered
towards your grade for this laboratory assignment. In conclusion, here are some
points to remember for creating programs that performs computations:

\begin{enumerate}

\item The provided source code contains many defects in it---make sure that you find them all!

\item The provide source code is incomplete---make sure that you add all of the needed features.

\item You should should draw a technical diagram to show the relationships between Java classes.

\item Make sure that you review the textbook's diagrams that explain parameter
  passing in Java.

\item You should think carefully about how to correctly use formal and actual parameters.

\item As in past assignments, your program only needs to have one {\tt main} method in one file.

\item Since your program will need to use different data types, please learn more about them.

\item Your program may need to read in additional values from the provided text file.

\item Your program will alternate between creating and displaying textual
  output---this is okay!

\item Don't forget to review the assignment sheets from the previous laboratory
  and practical assignments as they contain insights that will support your
  completion of this assignment.

\end{enumerate}

\section*{Summary of the Required Deliverables}

\noindent Students do not need to submit printed source code or technical
writing for any assignment in this course. Instead, this assignment invites you
to submit, using GitHub, the following deliverables. Please make sure that your
submitted source code does not contain any defects in it! Finally, you should
use comments document each fix that you introduced into the provided Java
classes.

\begin{enumerate}

  \setlength{\itemsep}{0in}

\item Stored in \reflection{}, a multiple-paragraph reflection on the commands
  that you typed in a text editor and the terminal window. This Markdown-based
  document should explain the input, output, and behavior of each command and
  the challenges that you confronted when using it. For every challenge that you
  encountered, please explain your solution for it. This document should also
  explain how you fixed at least one defect in the source code.

\item A complete and correct version of \mainprogramsource{} that meets all of
  the set requirements and produces the desired textual output in the terminal.

\item A complete and correct version of \secondprogramsource{} that meets all of
  the set requirements and supports the desired textual output in the terminal.

\end{enumerate}

\section*{Evaluation of Your Laboratory Assignment}

Using a report that the instructor shares with you through the commit log in
GitHub, you will privately received a grade on this assignment and feedback on
your submitted deliverables. Your grade for the assignment will be a function of
the whether or not it was submitted in a timely fashion and if your program
received a green \checkmark{} indicating that it met all of the requirements.
Other factors will also influence your final grade on the assignment. In
addition to studying the efficiency and effectiveness and documentation of your
Java source code, the instructor will also evaluate the correctness of your
technical writing. If your submission receives a red \naughtmark{}, the
instructor will reduce your grade for the assignment. Finally, please remember
to read your GitHub repository's \program{README.md} file for a description of
the four grades that you will receive for this laboratory assignment.

\section*{Adhering to the Honor Code}

The Academic Honor Program that governs the entire academic program at Allegheny
College is described in the Allegheny Academic Bulletin. The Honor Program
applies to all work that is submitted for academic credit or to meet non-credit
requirements for graduation at Allegheny College. This includes all work
assigned for this class (e.g., examinations, laboratory assignments, and the
final project). All students who have enrolled in the College will work under
the Honor Program. Each student who has matriculated at the College has
acknowledged the following pledge:

% \vspace*{-.11in}
\begin{quote}
  I hereby recognize and pledge to fulfill my responsibilities, as defined in
  the Honor Code, and to maintain the integrity of both myself and the College
  community as a whole.
\end{quote}
% \vspace*{-.11in}

\noindent It is understood that an important part of the learning process in any
course, and particularly one in computer science, derives from thoughtful
discussions with teachers and fellow students. Such dialogue is encouraged.
However, it is necessary to distinguish carefully between the student who
discusses the principles underlying a problem with others and the student who
produces assignments that are identical to, or merely variations on, someone
else's work. While it is acceptable for students in this class to discuss their
programs, data sets, and reports with their classmates, deliverables that are
nearly identical to the work of others will be taken as evidence of violating
the \mbox{Honor Code}. You must uphold the Honor Code while completing this
assignment. Students who do not understand how to adhere to the Honor Code
should talk with the course instructor during office hours.

\end{document}
