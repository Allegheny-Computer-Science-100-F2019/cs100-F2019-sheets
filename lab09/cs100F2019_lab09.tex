\documentclass[11pt]{article}

% NOTE: The "Edit" sections are changed for each assignment

% Edit these commands for each assignment

\newcommand{\assignmentduedate}{December 3}
\newcommand{\assignmentassignedate}{November 20}
\newcommand{\assignmentnumber}{Nine}

\newcommand{\labyear}{2019}
\newcommand{\labday}{Tuesday}
\newcommand{\labdueday}{Tuesday}
\newcommand{\labtime}{2:30 pm}

\newcommand{\assigneddate}{Assigned: \labday, \assignmentassignedate, \labyear{} at \labtime{}}
\newcommand{\duedate}{Due: \labdueday, \assignmentduedate, \labyear{} at \labtime{}}

% Edit these commands to give the name to the main program

\newcommand{\mainprogram}{\lstinline{TodoListManager}}
\newcommand{\mainprogramsource}{\lstinline{src/main/java/labnine/TodoListManager.java}}

\newcommand{\secondprogram}{\lstinline{TodoList}}
\newcommand{\secondprogramsource}{\lstinline{src/main/java/labnine/TodoList.java}}

\newcommand{\thirdprogram}{\lstinline{TodoListItem}}
\newcommand{\thirdprogramsource}{\lstinline{src/main/java/labnine/TodoListItem.java}}

% Edit this commands to describe key deliverables

\newcommand{\reflection}{\lstinline{writing/reflection.md}}
\newcommand{\readme}{\lstinline{README.md}}

% Commands to describe key development tasks

% --> Running gatorgrader.sh
\newcommand{\gatorgraderstart}{\command{gradle grade}}
\newcommand{\gatorgradercheck}{\command{gradle grade}}

% --> Compiling and running program with gradle
\newcommand{\gradlebuild}{\command{gradle build}}
\newcommand{\gradlerun}{\command{gradle run}}

% Commands to describe key git tasks

\newcommand{\gitcommitfile}[1]{\command{git commit #1}}
\newcommand{\gitaddfile}[1]{\command{git add #1}}

\newcommand{\gitadd}{\command{git add}}
\newcommand{\gitcommit}{\command{git commit}}
\newcommand{\gitpush}{\command{git push}}
\newcommand{\gitpull}{\command{git pull}}

\newcommand{\gitaddmainprogram}{\command{git add src/main/java/labnine/TodoListManager.java}}
\newcommand{\gitcommitmainprogram}{\command{git commit src/main/java/labnine/TodoListManager.java -m "Your
descriptive commit message"}}

% Use this when displaying a new command

\newcommand{\command}[1]{``\lstinline{#1}''}
\newcommand{\program}[1]{\lstinline{#1}}
\newcommand{\url}[1]{\lstinline{#1}}
\newcommand{\channel}[1]{\lstinline{#1}}
\newcommand{\option}[1]{``{#1}''}
\newcommand{\step}[1]{``{#1}''}

\usepackage{pifont}
\newcommand{\checkmark}{\ding{51}}
\newcommand{\naughtmark}{\ding{55}}

\usepackage{listings}
\lstset{
  basicstyle=\small\ttfamily,
  columns=flexible,
  breaklines=true
}

\usepackage{fancyhdr}
\usepackage{fancyvrb}

\usepackage[margin=1in]{geometry}
\usepackage{fancyhdr}

\pagestyle{fancy}

\fancyhf{}
\rhead{Computer Science 100}
\lhead{Laboratory Assignment \assignmentnumber{}}
\rfoot{Page \thepage}
\lfoot{\duedate}

\usepackage{titlesec}
\titlespacing\section{0pt}{6pt plus 4pt minus 2pt}{4pt plus 2pt minus 2pt}

\newcommand{\labtitle}[1]
{
  \begin{center}
    \begin{center}
      \bf
      CMPSC 100\\Computational Expression\\
      Fall 2019\\
      \medskip
    \end{center}
    \bf
    #1
  \end{center}
}

\begin{document}

\thispagestyle{empty}

\labtitle{Laboratory \assignmentnumber{} \\ \assigneddate{} \\ \duedate{}}

\section*{Objectives}

To further improve your expertise with designing, implementing, and enhancing
Java methods, including the completion of tasks such as creating and calling
methods that use boolean expressions, complex conditional logic, and iteration
constructs. You will learn how to use the \program{ArrayList} and the
\program{Iterator} to solve a real-world problem. Additionally, you will
practice using Java methods that read input from a text file with the
\program{java.util.Scanner}. Finally, to fully design and implement a real-world
program that provides useful features and to consider releasing it through
GitHub.

% REMINDER: Not doing teamwork until GatorGrader has new features
% and there is a polished version of the GatorGrouper tool

% Since you will complete this assignment with a partner, the last objective for
% this project is to further hone your team-work skills and better your ability to
% use a shared GitHub repository.

\section*{Suggestions for Success}

\begin{itemize}
  \setlength{\itemsep}{0pt}

\item {\bf Follow each step carefully}. Slowly read each sentence in the
  assignment sheet, making sure that you precisely follow each instruction. Take
  notes about each step that you attempt, recording your questions and ideas and
  the challenges that you faced. If you are stuck, then please tell a technical
  leader or instructor what assignment step you recently completed.

\item {\bf Regularly ask and answer questions}. Please log into Slack at the
  start of a laboratory or practical session and then join the appropriate
  channel. If you have a question about one of the steps in an assignment, then
  you can post it to the designated channel. Or, you can ask a student sitting
  next to you or talk with a technical leader or the course instructor.

\item {\bf Store your files in GitHub}. Starting with this laboratory
  assignment, you will be responsible for storing all of your files (e.g., Java
  source code and Markdown-based writing) in a Git repository using GitHub
  Classroom. Please verify that you have saved your source code in your Git
  repository by using \command{git status} to ensure that everything is
  updated. You can see if your assignment submission meets the established
  correctness requirements by using the provided checking tools on your local
  computer and by checking the commits in GitHub.

\item {\bf Keep all of your files}. Don't delete your programs, output files,
  and written reports after you submit them through GitHub; you will need them
  again when you study for the quizzes and examinations and work on the other
  laboratory, practical, and final project assignments.

\item {\bf Explore teamwork and technologies}. While certain aspects of the
  laboratory assignments will be challenging for you, each part is designed to
  give you the opportunity to learn both fundamental concepts in the field of
  computer science and explore advanced technologies that are commonly employed
  at a wide variety of companies. To explore and develop new ideas, you should
  regularly communicate with your team and/or the student technical leaders.

\item {\bf Hone your technical writing skills}. Computer science assignments
  require to you write technical documentation and descriptions of your
  experiences when completing each task. Take extra care to ensure that your
  writing is interesting and both grammatically and technically correct,
  remembering that computer scientists must effectively communicate and
  collaborate with their team members and the student technical leaders and
  course instructor.

\item {\bf Review the Honor Code on the syllabus}. While you may discuss your
  assignments with others, copying source code or writing is a violation of
  Allegheny College's Honor Code.

\end{itemize}

\section*{Reading Assignment}

To continue to learn more about ``{\tt if/else-if/else}'' statements and boolean
expressions, please again review Sections 5.1--5.3. Since this assignment will
also require you to continue to use Java classes and methods, you should once
again review Sections 4.1--4.5. To best prepare for the content in this
laboratory assignment, you should also study Sections 5.4--5.6, paying
particularly close attention to the material about {\tt while} loops, {\tt
break} and {\tt continue} statements, text file input, {\tt Iterators}, and the
{\tt java.util.ArrayList} class. Students who are not familiar with text-based
todo list management tools are encouraged to review the web site
\url{http://todotxt.org/}.

\section*{Accessing the Laboratory Assignment on GitHub}

To access the laboratory assignment, you should go into the
\channel{\#announcements} channel in our Slack workspace and find the
announcement that provides a link for it. Copy this link and paste it into your
web browser. Now, you should accept the laboratory assignment and see that
GitHub Classroom created a new GitHub repository for you to access the
assignment's starting materials and to store the completed version of your
assignment. Specifically, to access your new GitHub repository for this
assignment, please click the green ``Accept'' button and then click the link
that is prefaced with the label ``Your assignment has been created here''. If
you accepted the assignment and correctly followed these steps, you should have
created a GitHub repository with a name like
``Allegheny-Computer-Science-100-Fall-2019/computer-science-100-fall-2019-lab-9-gkapfham''.
Unless you provide the instructor with documentation of the extenuating
circumstances that you are facing, not accepting the assignment means that you
automatically receive a failing grade for it.

% To access the laboratory assignment, you should go into the
% \channel{\#announcements} channel in our Slack team and find the announcement
% that provides a link for it. Now, make sure that the leader of your team also
% notes your team number and first copies this link and pastes it into their web
% browser. Next, the team leader (i.e., the first person in the group listing on
% Slack) will create their team with the name
% \command{``Allegheny-Computer-Science-100-Fall-2019/computer-science-100-fall-2019-lab-9-gkapfham}
% and then accept the laboratory assignment and see that GitHub Classroom created
% a new GitHub repository containing the starting materials and to store the
% completed version of your assignment. Unless you provide the instructor with
% documentation of the extenuating circumstances that you are facing, not
% accepting the assignment means that you automatically receive a failing grade
% for it. Please see the course instructor with any questions.

% Note that the team leader will have to type their group and lab details into a
% text field. For instance, if the team leader was in the second group then that
% person would type ``Group 2 for Lab 9'' into the text field. At this point, each
% additional member of the team can accept the assignment through GitHub. Please
% make sure that each of your team members joins the team to which the instructor
% assigned them to work.

Before you move to the next step of this assignment, please make sure that you
read all of the content on the web site for your new GitHub repository, paying
close attention to the technical details about the commands that you will type
and the output that your program must produce. Now you are ready to download the
starting materials to your laboratory computer. Click the ``Clone or download''
button and, after ensuring that you have selected ``Clone with SSH'', please
copy this command to your clipboard. To enter the correct directory you should
now type \command{cd cs100F2019}. Next, you can type the command \command{ls}
and see that there are some files or directories inside of this directory. By
typing \command{git clone} in your terminal and then pasting in the string that
you copied from the GitHub site you will download all of the code for this
assignment. For the previous example, a student would run a \command{git clone}
command in the terminal in this fashion:

\begin{lstlisting}
  git clone git@github.com:Allegheny-Computer-Science-100-F2019/computer-science-100-fall-2019-lab-9-gkapfham.git
\end{lstlisting}

After this command finishes, you can use \command{cd} to change into the new
directory. If you want to \step{go back} one directory from your current
location, then you can type the command \command{cd ..}. Please continue to use
the \command{cd} and \command{ls} commands to explore the files that you
automatically downloaded from GitHub. What files and directories do you see?
What do you think is their purpose? Spend some time exploring, sharing your
discoveries with a neighbor and a \mbox{teaching assistant}.

\begin{figure}[tb]
  \begin{Verbatim}[commandchars=\\\{\}]

  Welcome to the Todo List Manager.
  What operation would you like to perform?
  Available options:read, priority-search, category-search, done, list, quit
  read
  list
  0, A, Understand, Draw diagram(s) to explain classes, done? false
  1, A, Understand, Use the LJV to see TodoList, done? false
  2, B, Explain, Add comments to all of the Todo classes, done? false
  priority-search
  What is the priority?
  A
  0, A, Understand, Draw diagram(s) to explain classes, done? false
  1, A, Understand, Use the LJV to see TodoList, done? false
  category-search
  What is the category?
  Explain
  2, B, Explain, Add comments to all of the Todo classes, done? false
  done
  What is the identifier of the task?
  0
  list
  0, A, Understand, Draw diagram(s) to explain classes, done? true
  1, A, Understand, Use the LJV to see TodoList, done? false
  2, B, Explain, Add comments to all of the Todo classes, done? false
  quit
  Thank you for using the Todo List Manager.
\end{Verbatim}
\vspace*{-.1in}
\caption{Sample ``{\tt TodoListManager}'' output featuring user input and tasks read from a file.}
\label{fig:output}
\end{figure}

\section*{Collaboratively Implementing a List Management Tool}

You should explore your repository by using a text editor to study the source
code of the {\tt TodoListManager.java}, {\tt TodoListItem.java}, and {\tt
TodoList.java} files. What methods do these classes provide? How do they work?
Can you draw a picture that depicts the relationship between these three Java
classes? Before you move onto the next step of this assignment, please look for
the \program{TODO} markers in the source code, making sure that you understand
what parts of the system are already implemented and what you have to extend and
add in order to complete the assignment.

You will notice that this assignment organizes the methods into three separate classes, as you have seen in past
assignments and in-class exercises. In particular, the {\tt TodoListManager} provides the user interface for our program
and the {\tt TodoList} furnishes the methods that perform the required computations and the {\tt TodoListItem}
represents an entry inside of a task list. If you want to make changes to the way in which the program accepts input or
processes variables to produce output, then you will need to modify the {\tt TodoListManager}. Otherwise, if you want to
modify the way in which the program performs a computation, or add a new computation, then you must make changes to
either the {\tt TodoList} or the {\tt TodoListItem}. Although more complex than this example, these three Java classes
complete their work by following a pattern similar to that which is in Figures 4.7 and 4.8. Please see the instructor if
you have questions about this approach.

The current implementation of the {\tt TodoListManager} reads from a file called ``{\tt todo.txt}'', an example of which
is included below this paragraph. An individual line in the ``{\tt todo.txt}'' file always adheres to the format
``Priority, Category, Task'' where ``{\tt A}'' is the most important priority (with ``{\tt B}'' being the next level,
and so on), ``{\tt Understand}'' being an example of a category, and ``{\tt Use the LJV to see the TodoList}'' is a
task.  Following this format for a task, you should consider adding in all of items that you and your partner must
complete in order to successfully finish this assignment. That is, you can actually use your {\tt TodoListManager} to
assist you as you complete both this project, later practical and laboratory assignments, and the upcoming final
project!

\vspace{-0.05in}
\begin{verbatim}
  A,Understand,Draw diagram(s) to explain classes
  A,Understand,Use the LJV to see TodoList
  B,Explain,Add comments to all of the Todo classes
\end{verbatim}
\vspace{-0.05in}

As shown in Figure~\ref{fig:output}, the finished version of the {\tt
TodoListManager} should provide features to read the todo list, search for
specific tasks according to both priority and category, mark a task as done, and
list all of the current tasks. Currently, the system does not include the source
code to implement the priority-search and category-search features. Yet, you can
see from the provided source code that the {\tt TodoListManager} can already
read the todo list from the file, mark a task as done, list the existing tasks,
and stop running the program. For this assignment, you must run the program by
typing \command{gradle -q --console plain run}. Can you build and run this
program?

To complete this assignment, you are responsible for adding all of the source
code that is needed to implement the priority-search and category-search
features. This means that you will first have to add code that can determine
when the user has input the word ``{\tt priority-search}'' or ``{\tt
category-search}''---what file should contain this code? Please notice that you
will need to finish implementing the methods that perform these operations! Both
of these operations will involve you using a {\tt java.util.Iterator} to iterate
through all of the instances of the {\tt TodoItem} class.

When you are performing a priority-search, you will need to collect and return
all of the {\tt TodoItem}s that match the provided priority level. For instance,
using the example todo list above on this page, a request for the ``{\tt A}''
priority tasks would return those with {\tt id} values of zero and one.
Similarly, the use of the category-search operation will require you to iterate
through all of the {\tt TodoItem}s managed by a {\tt TodoList} as you find those
that match the requested category. For the aforementioned list and a search for
the ``Explain'' category, the search would find only one task. The {\tt
  markTaskAsDone} method, as shown below, gives a concrete example of how to
  iterate through the {\tt todoItems} and use conditional logic to check if a
  specific {\tt todoItem} has the requested {\tt toMarkId}. You can use this
  method as an inspiration for those methods that you must implement.

\vspace{-0.05in}
\begin{verbatim}
public void markTaskAsDone(int toMarkId) {
    Iterator<TodoListItem> iterator = todoItems.iterator();
    while (iterator.hasNext()) {
      TodoListItem todoItem = iterator.next();
      if (todoItem.getId() == toMarkId) {
        todoItem.markDone();
      }
    }
  }
\end{verbatim}

\section*{Checking the Correctness of Your Program and Writing}

As verified by Checkstyle and GatorGrader, the source code for all three of your
files must adhere to all of the requirements in the Google Java Style Guide
available at \url{https://google.github.io/styleguide/javaguide.html} and in the
\readme{}. The Markdown file that contains your reflection must adhere to the
standards described in the Markdown Syntax Guide
\url{https://guides.github.com/features/mastering-markdown/}. Finally, your
\reflection{} file should adhere to the Markdown standards established by the
\step{Markdown linting} tool available at
\url{https://github.com/markdownlint/markdownlint/} and the writing standards
set by the \step{prose linting} tool from \url{http://proselint.com/}. Instead
of requiring you to manually check that your deliverables adhere to these
industry-accepted standards, the GatorGrader and Gradle tools that you use in
this assignment makes it easy for you to automatically check if your submission
meets these well-established standards for correctness. Please see the
instructor if you have questions about GatorGrader.

Since this is not your first laboratory assignment, you will notice that some of
the provided source code does not contain all of the required comments. This
means that you will have to inspect the source code from previous laboratory and
practical assignments to review how to create the comments in the files.
Moreover, the provided source code is missing many of the lines that are needed
to pass the GatorGrader checks. Make sure that you review the requirements for
these Java source code files, as outlined in this section and the previous one.
You can also study the source code of all three of the Java files to learn more
about what you need to add to them. For instance, you will need to both add the
implementation of a method and the source code that calls the method. Don't
forget to look in your GitHub repository to learn about GatorGrader's checks!

To get started with the use of GatorGrader, type the command \gatorgraderstart{}
in your terminal. If your laboratory work does not meet all of the assignment's
requirements, then you will see a summary of the failing checks along with a
statement giving the percentage of checks that are currently passing. If you do
have mistakes in your assignment, then you will need to review GatorGrader's
output, find the mistake, and try to fix it. Once your program is building
correctly, fulfilling at least some of the assignment's requirements, you should
transfer your files to GitHub using the \gitcommit{} and \gitpush{} commands.
For example, if you want to signal that the \mainprogramsource{} file has been
changed and is ready for transfer to GitHub you would first type
\gitcommitmainprogram{} in your terminal, followed by typing \gitpush{} and then
checking to see that the transfer to GitHub was successful. Make sure that you
transfer your source code and technical writing to GitHub, as the instructor
cannot access your deliverables unless you run the \gitpush{} command. Don't
forget that there are three Java source code files for this laboratory
assignment and, along with your technical writing, you must commit and push all
of them to GitHub! Please see the instructor or a teaching assistant if you
cannot upload your deliverables.

After the course instructor enables \step{continuous integration} with a system
called Travis CI, when you use the \gitpush{} command to transfer your source
code to your GitHub repository, Travis CI will initialize a \step{build} of your
assignment, checking to see if it meets all of the requirements. If both your
source code and writing meet all of the established requirements, then you will
see a green \checkmark{} in the listing of commits in GitHub after awhile. If
your submission does not meet the requirements, a red \naughtmark{} will appear
instead. The instructor will reduce a student's grade for this assignment if the
red \naughtmark{} appears on the last commit in GitHub immediately before the
assignment's due date. Yet, if the green \checkmark{} appears on the last commit
in your GitHub repository, then you satisfied all of the main checks, thereby
allowing the course instructor to evaluate other aspects of your source code and
writing, as further described in the \step{Evaluation} section of this
assignment sheet. Unless you provide the instructor with documentation of the
extenuating circumstances that you are facing, no late work will be considered
towards your grade for this laboratory assignment.

\noindent
In conclusion, here are some points to remember when creating your program that manages a list:

\begin{enumerate}
  \setlength{\itemsep}{0pt}

\item Please think carefully about the role of the three Java classes provided
  in your GitHub repository. To ensure that you add the correct functionality to
  the appropriate Java class, you should draw a technical diagram to illustrate
  the relationship between these classes.

\item The provided source code is incomplete---make sure that you add all of the
  needed features.

\item As in past assignments, your program only needs to have one {\tt main}
  method in one file.

\item Make sure that you know which of your Java classes needs to contain {\tt
  println} statements.

\item Don't forget to review the assignment sheets from the previous laboratory
  and practical assignments as they contain insights that will support your
  completion of this assignment.

% Not going to have teamwork on this assignment. See earlier and later notes.

% \item Don't forget that you and your team member(s) should all contribute
  % evenly by typing in text and using commands, like \command{git push} and
  % \command{git pull}, to transfer files to and from GitHub.

\end{enumerate}

Finally, for those students who are interesting in learning more about how the
Gradle build system runs this program, please study the source code of the
\program{build.gradle} file. As you reiew this file, please try to find the
following lines of source code and then understand how they enable you to run
the program interactively with the command \command{gradle -q --console plain
run}.

\begin{verbatim}
  // give the name of the application to run with "gradle run" command
  apply plugin: 'application'
  mainClassName = 'labnine.TodoListManager'

  // specify that the application will accept input from System.in
  run {
    standardInput = System.in
  }
\end{verbatim}

\section*{Summary of the Required Deliverables}

\noindent Students do not need to submit printed source code or technical
writing for any assignment in this course. Instead, this assignment invites you
to submit, using GitHub, the following deliverables.

\begin{enumerate}

  \setlength{\itemsep}{0in}

\item Stored in \reflection{}, a one-paragraph reflection on the commands that
  you typed in a text editor and the terminal. This Markdown-based document
  should explain the input, output, and behavior of each command and the
  challenges that you confronted when using it. For every challenge that you
  encountered, please explain your solution for it.

\item A complete and correct version of \mainprogramsource{} that meets all of
  the set requirements and produces the desired textual output in the terminal.

\item A complete and correct version of \secondprogramsource{} that meets all of
  the set requirements and supports the desired textual output in the terminal.

\item A complete and correct version of \thirdprogramsource{} that meets all of
  the set requirements and supports the desired textual output in the terminal.

% \item A commit log in your GitHub repository that clearly shows that the team
%   members effectively collaborated. That is, the commit log should show that
%   commits were evenly made by all team members; the photograph of each member
%   should appear in the commit log.

\end{enumerate}

\section*{Evaluation of Your Laboratory Assignment}

Using a report that the instructor shares with you through the commit log in
GitHub, you will privately received a grade on this assignment and feedback on
your submitted deliverables. Your grade for the assignment will be a function of
the whether or not it was submitted in a timely fashion and if your program
received a green \checkmark{} indicating that it met all of the requirements.
Other factors will also influence your final grade on the assignment. In
addition to studying the efficiency and effectiveness and documentation of your
Java source code, the instructor will also evaluate the correctness of your
technical writing. If your submission receives a red \naughtmark{}, the
instructor will reduce your grade for the assignment. Finally, please remember
to read your GitHub repository's \program{README.md} file for a description of
the four grades that you will receive for this laboratory assignment.

% \section*{Adhering to the Honor Code}

It is understood that an important part of the learning process in any course,
and particularly one in computer science, derives from thoughtful discussions
with teachers and fellow students. Such dialogue is encouraged. However, it is
necessary to distinguish carefully between the student who discusses the
principles underlying a problem with others and the student who produces
assignments that are identical to, or merely variations on, someone else's work.
While it is acceptable for students in this class to discuss their programs,
data sets, and reports with their classmates, deliverables that are nearly
identical to the work of others will be taken as evidence of violating the
\mbox{Honor Code}. You must uphold the Honor Code while completing this
assignment. Students who do not understand how to adhere to the Honor Code
should talk with the course instructor during office hours.

% See note at the start of the file.
% Remember that lab8 should also be a team-based
% assignment once issues about group work are fully resolved

% All team members will receive the same baseline grade for the laboratory
% assignment. If there are extenuating circumstances in which one or more of
% the team members do not effectively collaborate to complete this assignment,
% then the course instructor will adjust the grade of specific team members so
% that it is higher or lower than the baseline grade, as is fair and necessary.
% Please see the instructor if you do not understand how he assigns grades for
% collaborative assignments. Finally, in adherence to the Honor Code, students
% should only complete this assignment with their team members. Deliverables
% (e.g., Java source code or Markdown-based technical writing) that are nearly
% identical to the work of outsiders of your team will be taken as evidence of
% violating the \mbox{Honor Code}. Please see the course instructor with any
% questions about this assignment policy.

\end{document}
