\documentclass[11pt]{article}

% NOTE: The "Edit" sections are changed for each assignment

% Edit these commands for each assignment

\newcommand{\assignmentduedate}{October 7}
\newcommand{\assignmentassignedate}{October 4}
\newcommand{\assignmentnumber}{Five}

\newcommand{\labyear}{2019}
\newcommand{\labdueday}{Monday}
\newcommand{\labassignday}{Friday}
\newcommand{\labtime}{1:30 pm}

\newcommand{\assigneddate}{Assigned: \labassignday, \assignmentassignedate, \labyear{} at \labtime{}}
\newcommand{\duedate}{Due: \labdueday, \assignmentduedate, \labyear{} at \labtime{}}

% Edit these commands to give the name to the main program

\newcommand{\mainprogram}{\lstinline{PerformRedaction}}
\newcommand{\mainprogramsource}{\lstinline{src/main/java/practicalfour/PerformRedaction.java}}

% Edit this commands to describe key deliverables

\newcommand{\reflection}{\lstinline{writing/reflection.md}}

% Commands to describe key development tasks

% --> Running gatorgrader.sh
\newcommand{\gatorgraderstart}{\command{gradle grade}}
\newcommand{\gatorgradercheck}{\command{gradle grade}}

% --> Compiling and running program with gradle
\newcommand{\gradlebuild}{\command{gradle build}}
\newcommand{\gradlerun}{\command{gradle run}}

% Commands to describe key git tasks

% NOTE: Could be improved, problems due to nesting

\newcommand{\gitcommitfile}[1]{\command{git commit #1}}
\newcommand{\gitaddfile}[1]{\command{git add #1}}

\newcommand{\gitadd}{\command{git add}}
\newcommand{\gitcommit}{\command{git commit}}
\newcommand{\gitpush}{\command{git push}}
\newcommand{\gitpull}{\command{git pull}}

\newcommand{\gitcommitmainprogram}{\command{git commit src/main/java/practicalfour/PerformRedaction.java -m "Your
descriptive commit message"}}

% Use this when displaying a new command

\newcommand{\command}[1]{``\lstinline{#1}''}
\newcommand{\program}[1]{\lstinline{#1}}
\newcommand{\url}[1]{\lstinline{#1}}
\newcommand{\channel}[1]{\lstinline{#1}}
\newcommand{\option}[1]{``{#1}''}
\newcommand{\step}[1]{``{#1}''}

\usepackage{pifont}
\newcommand{\checkmark}{\ding{51}}
\newcommand{\naughtmark}{\ding{55}}

\usepackage{listings}
\lstset{
  basicstyle=\small\ttfamily,
  columns=flexible,
  breaklines=true
}

\usepackage{fancyvrb}
\usepackage{color}

\usepackage{fancyhdr}

\usepackage[margin=1in]{geometry}
\usepackage{fancyhdr}

\pagestyle{fancy}

\fancyhf{}
\rhead{Computer Science 100}
\lhead{Practical Assignment \assignmentnumber{}}
\rfoot{Page \thepage}
\lfoot{\duedate}

\usepackage{titlesec}
\titlespacing\section{0pt}{6pt plus 4pt minus 2pt}{4pt plus 2pt minus 2pt}

\newcommand{\labtitle}[1]
{
  \begin{center}
    \begin{center}
      \bf
      CMPSC 100\\Computational Expression\\
      Fall 2019\\
      \medskip
    \end{center}
    \bf
    #1
  \end{center}
}

\begin{document}

\thispagestyle{empty}

\labtitle{Practical \assignmentnumber{} \\ \assigneddate{} \\ \duedate{}}

\section*{Objectives}

To continue practicing the use of GitHub to access the files for a practical
assignment. Additionally, to practice using your laptop's operating system and
software development programs such as Docker Desktop, a terminal window, and a
text editor. You will continue to practice using Slack to support communication
with the teaching assistants and the course instructor. Next, you will learn
more about file input and output, further discovering how the course's automated
grading tool assesses your progress towards correctly completing the project.
Finally, you will learn more about how to use string replacement in a Java
program to ``redact'' text from the input file.

\section*{Suggestions for Success}

\begin{itemize}
  \setlength{\itemsep}{0pt}

\item {\bf Follow each step carefully}. Slowly read each sentence in the
  assignment sheet, making sure that you precisely follow each instruction.
  Take notes about each step that you attempt, recording your questions and
  ideas and the challenges that you faced. If you are stuck, then please tell a
  teaching assistant or instructor what assignment step you recently completed.

\item {\bf Regularly ask and answer questions}. Please log into Slack at the
  start of a laboratory or practical session and then join the appropriate
  channel. If you have a question about one of the steps in an assignment, then
  you can post it to the designated channel. Or, you can ask a student sitting
  next to you or talk with a teaching assistant or the course instructor.

\item {\bf Store your files in GitHub}. As in all of your past assignments, you
  will be responsible for storing all of your files (e.g., Java source code and
  Markdown-based writing) in a Git repository using GitHub Classroom. Please
  verify that you have saved your source code in your Git repository by using
  \command{git status} to ensure that everything is updated. You can see if
  your assignment submission meets the established correctness requirements by
  using the provided checking tools on your local computer and in checking the
  commits in GitHub.

\item {\bf Keep all of your files}. Don't delete your programs, output files,
  and written reports after you submit them through GitHub; you will need them
  again when you study for the quizzes and examinations and work on the other
  laboratory, practical, and final project assignments.

\item {\bf Explore teamwork and technologies}. While certain aspects of these
  assignments will be challenging for you, each part is designed to give you the
  opportunity to learn both fundamental concepts in the field of computer
  science and explore advanced technologies that are commonly employed at a wide
  variety of companies. To explore and develop new ideas, you should regularly
  communicate with your team and/or the student technical leaders.

\item {\bf Hone your technical writing skills}. Computer science assignments
  require to you write technical documentation and descriptions of your
  experiences when completing each task. Take extra care to ensure that your
  writing is interesting and both grammatically and technically correct,
  remembering that computer scientists must effectively communicate and
  collaborate with their team members and the tutors, teaching assistants, and
  course instructor.

\item {\bf Review the Honor Code on the syllabus}. While you may discuss your
  assignments with others, copying source code or writing is a violation of
  Allegheny College's Honor Code.

\end{itemize}

\section*{Reading Assignment}

If you have not done so already, please read all of the relevant ``GitHub
Guides'', available at \url{https://guides.github.com/}, that explain how to use
many of the features that GitHub provides. In particular, please make sure that
you have read guides such as ``Mastering Markdown'' and ``Documenting Your
Projects on GitHub''; each of them will help you to understand how to use both
GitHub and GitHub Classroom. Focusing on the content about declaring and
variables and reading from files, you should review Chapters 1 through 3 and
textbook, paying close attention to the technical content about string
manipulation through the use of methods like \command{replace}.

\section*{Performing Text Redaction Through String Manipulation}

To access the practical assignment, you should go into the
\channel{\#announcements} channel in our Slack team and find the announcement
that provides a link for it. Copy this link and paste it into your web browser.
Now, you should accept the practical assignment and see that GitHub Classroom
created a new GitHub repository for you to access the assignment's starting
materials and to store the completed version of your assignment. Specifically,
to access your new GitHub repository for this assignment, please click the green
``Accept'' button and then click the link that is prefaced with the label ``Your
assignment has been created here''. If you accepted the assignment and correctly
followed these steps, you should have created a GitHub repository with a name
like
``Allegheny-Computer-Science-100-Fall-2019/computer-science-100-fall-2019-practical-4-gkapfham''.
Unless you provide the instructor with documentation of the extenuating
circumstances that you are facing, not accepting the assignment means that you
automatically receive a failing grade for it. Please follow the steps from the
previous laboratory assignments for finding your ``home base'' for this
practical assignment; see the instructor if you are stuck on getting started.

Figure~\ref{mad} contains the output from running a program like the one you
must implement. To ensure that the program produces the correct output, you
should follow the instructions at all of the \command{TODO} markers, adding the
needed source code. As you are working on this practical assignment, make sure
that you study the following source code segment and understand how it writes
the \command{redactedSentence} variable to the output file. If your program
works correctly, then you should see that it produces the file
\program{content/sentence\_output.txt}. If your Java program does not produce
this file or the file does not have the correct contents, then make sure that
you talk with the course instructor or a student technical leader to identify
and fix any mistakes in your source code.

\begin{verbatim}
    try {
      PrintWriter out = new PrintWriter("content/sentence_output.txt");
      out.println(redactedSentence);
      out.flush();
      out.close();
    } catch (FileNotFoundException noFile) {
      System.out.println("Unable to locate the input file");
    }
\end{verbatim}

\begin{figure}[tb]
\begin{Verbatim}[commandchars=\\\{\}]
  Gregory M. Kapfhammer Fri Oct 04 08:03:13 EDT 2019

  Performing redaction on the following sentence:
  I am a student enrolled in Computer Science 100 at Allegheny College.

  Redaction created the following sentence:
  I am a student enrolled in Computer Science 100 at GGGGGGGGG College.

  Thanks for using the redaction program.
  Have an awesome day.
\end{Verbatim}
\vspace*{-.1in}
\caption{Sample Output from Running \mainprogram.}
\label{mad}
\end{figure}

\section*{Checking the Correctness of Your Program and Writing}

As in the past assignments, you are provided with an automated tool for checking
the quality of your source code. Please note that the practical assignments do
not require you to produce a writing document as you do in the laboratory
assignments. However, to check your Java source code you can started with the
use of GatorGrader, type the command \gatorgraderstart{} in your Docker
container. If your work does not meet all of the assignment's requirements, then
you will see output that highlights the mistakes that you made. If you do have
mistakes in your assignment, then you will need to review GatorGrader's output,
find the mistake, and try to fix it. Specifically, don't forget to add in the
required comments and commit to your repository a sufficient number of times!
Finally, don't forget to ensure that you still transfer all of your source code
to your GitHub repository. Please see the course instructor if you have
questions about this important step.

Once your program is building correctly, fulfilling at least some of the
assignment's requirements, you should transfer your files to GitHub using the
\gitcommit{} and \gitpush{} commands. For example, if you want to signal that
the \mainprogramsource{} file has been changed and is ready for transfer to
GitHub you would first type \gitcommitmainprogram{} in your terminal, followed
by typing \gitpush{} and checking to see that the transfer to GitHub is
successful. If you notice that transferring your code to GitHub did not work
correctly, then please try to determine why, asking a technical leader or the
course instructor for help, if necessary.

After the course instructor enables \step{continuous integration} with a system
called Travis CI, when you use the \gitpush{} command to transfer your source
code to your GitHub repository, Travis CI will initialize a \step{build} of
your assignment, checking to see if it meets all of the requirements. If both
your source code and writing meet all of the established requirements, then you
will see a green \checkmark{} in the listing of commits in GitHub after awhile.
If your submission does not meet the requirements, a red \naughtmark{} will
appear instead. The instructor will reduce a student's grade for this
assignment if the red \naughtmark{} appears on the last commit in GitHub
immediately before the assignment's due date. Yet, if the green \checkmark{}
appears on the last commit in your GitHub repository, then you satisfied all of
the main checks. Unless you provide the course instructor with documentation of
the extenuating circumstances that you are facing, no late work will be
considered towards your completion grade for this practical assignment. You
should aim to finish practical assignments on the day that they are assigned,
finishing them by the start of the next classroom session.

% Give additional suggestions for improvement, adding to the first list

\section*{Additional Success Strategies for the Practical Sessions}

Since this is one of our first practical assignments and you are still learning
how to use the Java programming language, don't become frustrated if you make a
mistake. Instead, use your mistakes as an opportunity for learning both about
the necessary technology and the background and expertise of the other students
in the class, the technical leaders, and the course instructor.

\vspace*{-.05in}
\begin{itemize}

\itemsep 0in

\item {\bf Experiment.} Practical sessions are for learning by doing without the
  pressure of grades or ``right/wrong'' answers. So try things! To learn about
  files and variables, please intelligently experiment with the code, making
  small incremental changes and observing the output.

\item {\bf Practice Key Laboratory Skills.} As you are completing this
  assignment, practice using your text editor and the terminal until you can
  easily use their most important features. If you would like, you can also try
  out other text editors and advanced terminal commands.

\item {\bf Help One Another!} If your neighbor is struggling and you know what
  to do, offer your help. Don't ``do the work'' for them, but advise them on
  what to type or how to handle things. If you are stuck on a part of this
  practical assignment and you could not find any insights in either your
  textbook or online sources, formulate a question to ask your neighbor, a
  teaching assistant, or a course instructor. Try to strike the right balance
  between asking for help when you cannot solve a problem and working
  independently to find a solution.

\end{itemize}

\section*{Summary of the Required Deliverables}

\noindent Students do not need to submit printed source code or technical
writing for any assignment in this course. Instead, this assignment invites you
to submit, using GitHub, the following deliverables. Because this is a practical
assignment, you are not required to complete any technical writing.

\begin{enumerate}

\setlength{\itemsep}{0in}

\item A correct version of \mainprogramsource{} that meets all of the
  established source code requirements and produces the desired text-based
  output.

\end{enumerate}

\section*{Evaluation of Your Practical Assignment}

Practical assignments are graded on a completion --- or ``checkmark'' --- basis.
If your GitHub repository has a \checkmark{} for the last commit before the
deadline then you will receive the highest possible grade for the assignment.
However, you will fail the assignment if you do not complete it correctly, as
evidenced by a red \naughtmark{} in your commit listing, by the set deadline for
finishing the project.

\section*{Adhering to the Honor Code}

In adherence to the Honor Code, students should complete this assignment on an
individual basis. While it is appropriate for students in this class to have
high-level conversations about the assignment, it is necessary to distinguish
carefully between the student who discusses the principles underlying a problem
with others and the student who produces assignments that are identical to, or
merely variations on, someone else's work. Deliverables (e.g., Java source code
or Markdown-based technical writing) that are nearly identical to the work of
others will be taken as evidence of violating the \mbox{Honor Code}. Please see
the course instructor if you have questions about this policy.

\end{document}
