\documentclass[11pt]{article}

% NOTE: The "Edit" sections are changed for each assignment

% Edit these commands for each assignment

\newcommand{\assignmentduedate}{December 13}
\newcommand{\assignmentassignedate}{November 22}
\newcommand{\assignmentnumber}{Final: Open-Source Software}

\newcommand{\labyear}{2019}
\newcommand{\labday}{Friday}
\newcommand{\labdueday}{Friday}
\newcommand{\labduetime}{12:00 midnight}
\newcommand{\labtime}{9:00 am}

\newcommand{\assigneddate}{Assigned: \labday, \assignmentassignedate, \labyear{} at \labtime{}}
\newcommand{\duedate}{Due: \labdueday, \assignmentduedate, \labyear{} at \labduetime{}}

% Edit these commands to give the name to the main program

\newcommand{\mainprogram}{\lstinline{DoublyLinkedList}}
\newcommand{\mainprogramsource}{\lstinline{src/main/java/practicaleight/list/DoublyLinkedList.java}}

% Edit these commands to give the main program's output details

\newcommand{\mainprogramoutput}{four}

% Edit these commands to give the name to the test suite

\newcommand{\testprogram}{\lstinline{TestDoublyLinkedList}}
\newcommand{\testprogramsource}{\lstinline{src/test/java/practicaleight/TestDoublyLinkedList.java}}

% Edit this commands to describe key deliverables

\newcommand{\reflection}{\lstinline{writing/reflection.md}}

% Commands to describe key development tasks

% --> Running gatorgrader.sh
\newcommand{\gatorgraderstart}{\command{gradle grade}}
\newcommand{\gatorgradercheck}{\command{gradle grade}}

% --> Compiling and running and testing program with gradle
\newcommand{\gradlebuild}{\command{gradle build}}
\newcommand{\gradletest}{\command{gradle test}}
\newcommand{\gradlerun}{\command{gradle run}}

% Commands to describe key git tasks

% NOTE: Could be improved, problems due to nesting

\newcommand{\gitcommitfile}[1]{\command{git commit #1}}
\newcommand{\gitaddfile}[1]{\command{git add #1}}

\newcommand{\gitadd}{\command{git add}}
\newcommand{\gitcommit}{\command{git commit}}
\newcommand{\gitpush}{\command{git push}}
\newcommand{\gitpull}{\command{git pull}}

\newcommand{\gitcommitmainprogram}{\command{git commit src/main/java/practicaleight/list/DoublyLinkedList.java -m "Your
descriptive commit message"}}

% Use this when displaying a new command

\newcommand{\command}[1]{``\lstinline{#1}''}
\newcommand{\program}[1]{\lstinline{#1}}
\newcommand{\url}[1]{\lstinline{#1}}
\newcommand{\channel}[1]{\lstinline{#1}}
\newcommand{\option}[1]{``{#1}''}
\newcommand{\step}[1]{``{#1}''}

\usepackage{pifont}
\newcommand{\checkmark}{\ding{51}}
\newcommand{\naughtmark}{\ding{55}}

\usepackage{listings}
\lstset{
  basicstyle=\small\ttfamily,
  columns=flexible,
  breaklines=true
}

\usepackage{fancyhdr}

\usepackage[margin=1in]{geometry}
\usepackage{fancyhdr}

\pagestyle{fancy}

\fancyhf{}
\rhead{Computer Science 100}
\lhead{Practical Assignment \assignmentnumber{}}
\rfoot{Page \thepage}
\lfoot{\duedate}

\usepackage{titlesec}
\titlespacing\section{0pt}{6pt plus 4pt minus 2pt}{4pt plus 2pt minus 2pt}

\newcommand{\labtitle}[1]
{
  \begin{center}
    \begin{center}
      \bf
      CMPSC 100\\Computational Expression\\
      Fall 2019\\
      \medskip
    \end{center}
    \bf
    #1
  \end{center}
}

\begin{document}

\thispagestyle{empty}

\labtitle{Practical \assignmentnumber{} \\ \assigneddate{} \\ \duedate{}}

\vspace*{-.2in}

\section*{Objectives}

To learn more about the world of open-source software and to practice the skill
of downloading, installing, and using open-source software written in a
programming language like Java.
%
You will learn more about a total of three open-source projects available on
GitHub and then attempt to download and install one of these projects,
ultimately reporting on your experience in through technical writing.
%
If possible, students should attempt to find GitHub-hosted projects that connect
to the course themes of algorithms, data structures, and experimental analysis.
%
With that said, you are allowed to review any three open-source projects that
are currently available for download from GitHub.
%
Finally, you should try to download, use, and install at least one of the three
projects. While earning full credit on this assignment does not require you to
successfully install and use an open-source project, students are encouraged to
strive towards the completion of this goal.
%
In addition to talking with the course instructor, students should refer to the
documentation in their GitHub repository for more details about the steps that
they must take to complete this project.

\section*{Accessing the Practical Assignment on GitHub}

To access the practical assignment, you should go into the
\channel{\#announcements} channel in our Slack team and find the announcement
that provides a link for it. If you accepted the assignment and correctly
followed the appropriate startup steps, you should have created a GitHub
repository with a name like
``Allegheny-Computer-Science-100-Fall-2019/computer-science-100-fall-2019-practical-final-gkapfham''.
%
Before you move to the next step of this assignment, please make sure that you
read all of the content on the web site for your new GitHub repository, paying
close attention to the technical details about the commands that you will type
and the output that your program must produce. Now you are ready to download the
starting materials to your practical computer. Click the ``Clone or download''
button and, after ensuring that you have selected ``Clone with SSH'', please
copy this command to your clipboard. Please note that there is no provided
source code for this assignment!
%
Instead, students are invited to write about three open-source projects
available on GitHub and to document their experiences with installing one of
these projects.

\section*{Summary of the Required Deliverables}

\noindent Students do not need to submit printed source code or technical
writing for any assignment in this course. Instead, this assignment invites you
to submit, using GitHub, the following deliverables.

\vspace*{-.05in}

\begin{enumerate}

  \setlength{\itemsep}{0in}

\item A document written in Markdown that provides, for instance, three
  paragraphs that give an overview of three separate open-source projects
  available for download from GitHub.

\end{enumerate}

\vspace*{-.1in}

\section*{Evaluation of Your Final Practical Assignment}

Using a report that the instructor shares with you through the commit log in
GitHub, you will privately received a grade on this assignment and feedback on
your submitted deliverables. Your grade for the assignment will be a function of
the whether or not it was submitted in a timely fashion and if your program
received a green \checkmark{} indicating that it met all of the baseline
requirements checked by GatorGrader.
%
Please remember to read your GitHub repository's \program{README.md} file for a
description of the completion grade that you will receive for this practical
assignment.
%
You can talk with the instructor if you have questions about the evaluation of
this practical assignment.

% \section*{Adhering to the Honor Code}

% In adherence to the Honor Code, students should complete this assignment on an
% individual basis. While it is appropriate for students in this class to have
% high-level conversations about the assignment, it is necessary to distinguish
% carefully between the student who discusses the principles underlying a problem
% with others and the student who produces assignments that are identical to, or
% merely variations on, someone else's work. Deliverables (e.g., Java source code
% or Markdown-based technical writing) that are nearly identical to the work of
% others will be taken as evidence of violating the \mbox{Honor Code}. Please see
% the course instructor if you have questions about this policy.

\end{document}
